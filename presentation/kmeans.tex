\begin{frame}
    \frametitle{$k$-Means Motivation}

    \begin{columns}
        \begin{column}{5cm}
           \begin{itemize}
               \item Often, you don't have much information about the 
                   structure of $X$.
               \item In fact, we did not use any in the PCA step.
               \item<2-> By visualization, you can guess structure in $X$,\\
                   \alert{``there might be 3 clusters''}.
           \end{itemize}
        \end{column}

        \begin{column}{5cm}
            \includegraphics<1>[width=\linewidth]{pca-pics/iris-all-nocolor}
            \includegraphics<2>[width=\linewidth]{pca-pics/iris-2d-nocolor}
        \end{column}
    \end{columns}

\end{frame}

\begin{frame}
    \frametitle{$k$-Means Questions}

    \begin{columns}
        \begin{column}{5cm}
            \begin{enumerate}[<+->]
               \item Can we \alert{assign} data points to clusters?
               \item Can we find a \alert{representative} for each cluster?
           \end{enumerate}
        \end{column}
        \begin{column}{5cm}
            \includegraphics<1->[width=\linewidth]{pca-pics/iris-2d-nocolor}
        \end{column}
    \end{columns}

\end{frame}

\begin{frame}
    \frametitle{$k$-Means Algorithm}
    $k$-Means finds assignments $j$ and cluster centers $\mu$ by solving
    \begin{align}
        \min_{\mu}\sum_{i=0}^{N} \min_j\left\|\mu_j-x_i\right\|^2
        \label{eq:km}
    \end{align}
    The algorithm is simple:
    \begin{enumerate}
        \item Set $\mu$, $j$ to a random value
        \item Solve \eqref{eq:km} for $j$
        \item Solve \eqref{eq:km} for $\mu$
        \item If $j$ or $\mu$ changed significantly, go to step 2.
    \end{enumerate}
\end{frame}

\begin{frame}
    \frametitle{$k$-Means Visualization}
    \begin{center}
    \href{http://icperformance.com/wp-content/demos/kmeansmouse.html}{K-Means 
        Website Example}
    \end{center}
\end{frame}

